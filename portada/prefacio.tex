{\Large \bfseries Resumen}

La adopción cada vez más amplia por parte de entidades públicas y privadas de los Datos Abiertos o \textbf{Open Data}, abre la puerta a nuevos proyectos y desarrollos que hacen uso de estos en múltiples campos y contextos. \textbf{FiWare}, es una plataforma abierta focalizada en la gestión y consumo de datos para Smart Cities, y mediante su marco de herramientas de desarrollo nos permite integrar estos datos abiertos en sistemas a gran escala. Desplegar y administrar estas aplicaciones y sistemas a nivel de producción, incluso para configuraciones aparentemente simples, puede introducir enormes barreras de complejidad para los desarrolladores y científicos de datos, tanto por la cantidad de datos en tiempo real a procesar como por la complejidad que pueden adquirir estos sistemas a la hora de escalarlos. \textbf{Kubernetes} y \textbf{Spark}, entre otras tecnologías, nos proporcionan esta capa de abstracción a la hora de pensar en estos sistemas a gran escala, y hacen más fácil y rápido el tratamiento y mantenimiento de ellos.

El principal objetivo de este proyecto es el de demostrar el potencial de estas tecnologías por separado y en conjunto, que nos permiten automatizar y mejorar el despliegue de este tipo de entornos. Para ello, desarrollaremos un caso de uso que involucre todo el proceso: una aplicación web para la interacción con el usuario, una arquitectura en la nube basada en micro servicios y un modelo de predicción entrenado con datos recogidos en tiempo real, que permitan al usuario obtener una predicción de la ocupación de los aparcamientos públicos de su ciudad.\\

{\Large \bfseries Abstract}

The increasing adoption of \textbf{Open Data} by public and private entities opens the door to new projects and developments that make use of these huge amounts of data for multiple applications in several fields and contexts. \textbf{FiWare} is an open platform focused on the management and consumption of data for Smart Cities, and through its framework of development tools, allows us to integrate these unstructured open data in large-scale systems. Deploying and administrating these applications and systems at production level, even for seemingly simple configurations, can introduce huge barriers of complexity for developers and data scientists, both because of the amount of real-time data to be processed and the increasing complexity that these systems can acquire when scaling them out. \textbf{Kubernetes} and \textbf{Spark}, among other technologies, provide us with this layer of abstraction when thinking about these large-scale systems, and make it easier and faster to process and maintain them.

The main goal of this project is to show the potential of these technologies both separately and together, allowing us to automate and improve the deployment of this kind of environment. To do so, we will design a use case that involves the whole process: a web application for user interaction, a cloud architecture based on micro-services, and a prediction model trained with open data, collected in real-time from a public data source, that will allow the user to obtain a prediction of the occupancy of the public parking spaces in his city on a given date and time.

\textit{\textbf{Palabras clave / Key words: } Open Data, Smart Cities, Ciudades Inteligentes, IoT, FIWARE, Spark, Kubernetes, Docker, Machine Learning, Predictive Analysis, Big Data. }