\chapter{Introduction}
\label{chapter:Introduction}

The idea of open data, and more precisely open government data, has been around for a few decades, but today, with the advent of the data revolution and the growing volume of data generated in all possible areas around us, exploiting them makes more sense than ever.

The fourth industrial revolution grounds on the data revolution, generating a global trend toward more data-driven decision-making in policy-making \cite{open-janssen}, but right now, open data,  and specifically open government data, is a tremendous resource that is still largely underused. The data generated by the administration is mostly public by law, and as every entity from local administrations to the national government are in charge of managing public infrastructure, the possibilities of capturing and processing this data are huge.

FIWARE is the response of the European Union researchers to face this challenge, and its goal is to advance the global competitiveness of the EU economy by introducing an innovative infrastructure for cost-effective creation and delivery of services, providing high QoS and security guarantees \cite{fiware}. This platform allows public administration and private entities to standardize the way they consume and expose their data, providing an open architecture with generic and reusable building blocks. A lot of use cases has already been proposed and developed under the Fiware standard, but in this project we want to combine some of the components provided with other tools usually found in cloud deployments, like Docker containers and the Kubernetes orchestation tools, and set up a whole cloud environment, including a machine learning model, to make predictions on the data collected.

The idea for this project is to design a scalable and failure resilient system, as besides being used for this specific use case, we want to create a reusable deployment, that can be applied in other solutions in the future with few modifications.  Kubernetes, MongoDB, and Spark are widely used open-source tools, used by some of the biggest technology companies of the world for solutions involving huge ingestions of data, and when using them as a base to build our system, we can make a really adaptable, scalable and failure-resistant system, even if we are over dimensioning the scope of this specific use case.

